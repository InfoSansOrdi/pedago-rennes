\documentclass[a4paper,10pt]{article}
\usepackage[utf8]{inputenc}
\usepackage{tikz}
\usepackage{amsmath}
\usepackage{fullpage}

%opening
\title{Initiations aux protocoles de sécurité}

\begin{document}

\maketitle


\section{But}
Faire prendre conscience aux joueurs de l'importance d'avoir des protocoles de sécurité sans failles.

\section{Principe de l'activité}
Les joueurs vont tester des protocoles fournis en exemple, leur but est de s'échanger un message en s'assurant que celui-ci reste secret, en utilisant les clés et les boîtes à leur disposition ; un des joueurs prend le rôle de l'attaquant et doit réussir à obtenir le message, en le sortant de la boîte dans laquelle il se trouve ; un des joueurs joue le rôle d'arbitre, c'est par lui que les joueurs passeront pour échanger des messages.

\section{Matériel}
\begin{itemize}
\item Des cartes Messages (en plusieurs exemplaires).
\item Des clés rondes colorées.
\item Des clés triangulaires uniques numérotées de $1$ à $p$.
\item Des clés carrées numérotées de $1$ à $p$.
\item Des boîtes verrouillées par des clés rondes colorées.
\item Des boîtes verrouillées par des clés triangulaires numérotées et ouvrables par les clés carrées correspondantes.
\item Des boîtes verrouillées par des clés carrées numérotées et ouvrables avec les clés triangulaires correspondantes.
\item Des masques cartonnés.
\item Des exemples de protocole.
\end{itemize}


\section{Règles}
\begin{itemize}
\item Les boîtes ne peuvent pas être cassées. Elles ne peuvent être ouvertes que si vous disposez de la bonne clé ; de même, vous ne pouvez verrouiller une boîte que si vous disposez de la bonne clé.
\item Pour donner un message à un joueur, donnez le à l'arbitre en lui pointant du doigt le joueur à qui vous voulez envoyer le message.
\item Lorsque l'arbitre reçoit un message, il doit immédiatement en faire une copie (il peut pour cela l'ouvrir sans avoir de clé, et c'est le seul joueur à posséder ce droit) et la donner à l'attaquant.
\item L'attaquant a le droit de forcer l'arbitre à lui donner un message, pour que celui-ci n'arrive pas à destination. Il a également le droit de donner à n'importe quel moment un message à l'arbitre en lui pointant du doigt le joueur auquel il souhaite l'envoyer. 
\end{itemize}

\end{document}
