\documentclass[12pt,landscape]{article}
\usepackage[a3paper, margin=0.5cm]{geometry}

\usepackage[french]{babel}
\usepackage[T1]{fontenc}

\author{Paul ADAM \and Jules TIMMERMAN}
\title{Fiche de préparation :\\Image et Compression}

\begin{document}

\begin{tabular}{|c|c|c|c|c|}
	\hline
	\textbf{Titre} & \multicolumn{4}{|c|}{Images et Compression}\\
	\hline
	% TODO : temps
	\textbf{Durée} & \multicolumn{4}{|c|}{55 minutes}\\
	\hline
	\textbf{Prérequis} & \multicolumn{4}{|c|}{Aucun}\\
	\hline
	\textbf{Description} & \multicolumn{4}{|c|}{L'objectif est de savoir traduire une image en nombres et inversement. De plus, on introduit un concept similaire au Run-length encoding pour compresser l'image}\\
	\hline
	\textbf{Matériel} & \multicolumn{4}{|c|}{Fiches avec des images}\\
	\hline
	\textbf{Durée} & \textbf{Phase} & \textbf{Activités} & \textbf{Orga} & \textbf{Matériel}\\
	\hline

	5' & Explication du but & Décrire des images & Tableau & Tableau\\
	5' & Travail & Ils tentent (idéalement en décrivant les formes) & Groupe de 2 & Fiches d'images\\
	5' & Propositions & Comment vous avez fait ? Élèves proposent comment faire. Décrire dans l'ordre (NNBNN...) avec initiale & Dialogue & Tableau\\
	8' & Travail & Description d'images de tests chacun leur tour. Ecrire sur l'ardoise. & Groupe de 2 & Fiche d'images\\
	5' & Amélioration & Long de décrire. Proposition ? Donner le nombre de répétitions (par ligne) & Dialogue & Tableau\\
	8' & Travail & Description avec la nouvelle méthode & Groupe de 2 & Fiche d'images\\
	5' & Amélioration & Encore meilleure idée ? Retour à la ligne et taille d'images & Dialogue & Tableau\\
	8' & Travail & Description avec la nouvelle nouvelle méthode & Groupe de 2 & Fiche d'images\\
	5' & InfoParceQue & cf plus bas & Tableau & Tableau\\


	\hline

	& \multicolumn{4}{|c|}{
		Traduire une info de manière à être compris (nombre pour les ordinateurs). Généralisable aux couleurs
	}\\
	\textbf{Info parce que} & \multicolumn{4}{|c|}{
		Utiliser le moins de place possible : compression, c'est important.
	}\\
	& \multicolumn{4}{|c|}{
		Vraiment utilisé à plusieurs endroits (avec des raffinements comme la répétition de séquences plus compliquées)
	}\\

	\hline

	\textbf{Remarques} & \multicolumn{4}{|c|}{
		Cette activité se prête très bien aussi à des avancées dépendantes par groupes plutôt qu'une avancée globale. On peut aussi laisser aux élèves proposer leurs propres images. 
	}\\
	
	
	\hline
\end{tabular}


\end{document}