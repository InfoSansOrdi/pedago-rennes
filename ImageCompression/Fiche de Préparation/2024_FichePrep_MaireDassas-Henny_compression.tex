\documentclass[12pt,landscape]{article}
\usepackage[a3paper, margin=0.5cm]{geometry}

\usepackage[french]{babel}
\usepackage[T1]{fontenc}

\author{Grégoire Maire \and Nicolas Dassas-Henny}
\title{Fiche de préparation :\\Images téléphonées}

\begin{document}

\begin{tabular}{|c|c|c|c|c|}
	\hline
	\textbf{Titre} & \multicolumn{4}{|c|}{Images téléphonés}\\
	\hline
	% TODO : temps
	\textbf{Durée} & \multicolumn{4}{|c|}{50 minutes +5/10 minutes de marge}\\
	\hline
	\textbf{Prérequis} & \multicolumn{4}{|c|}{Aucun}\\
	\hline
	\textbf{Description} & \multicolumn{4}{|c|}{Réussir à décrire clairement et rapidement une image, c'est à dire réussir à la compresser de manière non ambigüe.}\\
	\hline
	\textbf{Matériel} & \multicolumn{4}{|c|}{Fiches avec des images et grilles vides}\\
	\hline
	\textbf{Durée} & \textbf{Phase} & \textbf{Activités} & \textbf{Orga} & \textbf{Matériel}\\
	\hline

	5' & Explication du but & Essayer pour deux dessinateurs de reproduire exactement une image décrite par un orateur. & Tableau & Rien\\
	10' & Travail & Deux élèves par groupe tentent (idéalement en décrivant les formes) de reproduire l'image décrite (on fait tourner) & Groupe de 3 & 3 Images, 9 grilles $\infty$/grp \\
	5' & Propositions & Demander aux élèves comment ils ont fait. Expliquer rapidement qu'on n'a pas une place infini sur un ordi et qu'on aimerait  & Dialogue & Tableau\\ . & . &
    qu'ils décrivent plus rapidement, sans forcément utiliser des phrases. Si des élèves donnent& &\\ . & . & une bonne idée (NBNNN...), on peut demander à la classe quelle est selon eux la technique proposée la plus efficace.& &\\
	15' & Travail & Les élèves continuent de décrire en utilisant de nouvelles méthodes. & Groupe de 3 & 3 Images, 9 grilles $\infty$/grp\\
	avec phase précédente & Amélioration & Si les élèves n'ont pas trouvé par eux-mêmes, les guider vers NBNBNN puis N5B3N4. & Passage dans les rangs & Aucun\\
	avec phase précédente & Extension & On ne peut dire qu'une fois le nom d'une couleur (avec alternance) & Groupe de 3 & si besoin + de img, grilles\\
	5' & Travail & Distribution de parties d'une image commune à colorier (pour parler rapidement de parallélisme) puis remplissage& Groupe de 3 & Image commune\\
	5' & Remise en commun & Reveal image commune puis quelles ont été les méthodes utilisées par les enfants ? & Tableau & Tableau\\
	5' & InfoParceQue & cf plus bas (il y a également une trace écrite à donner au prof) & Tableau & Tableau\\


	\hline

	& \multicolumn{4}{|c|}{
		Compression nécessaire des images car mémoire limitée. Vraiment utilisé : PNG, c'est comme ça que c'est stocké sur vos ordis.
	}\\
	\textbf{Info parce que} & \multicolumn{4}{|c|}{
		Important de se mettre d'accord : protocole informatique.
	}\\
	& \multicolumn{4}{|c|}{
		Systèmes distribués (image commune): mise en commun du travail, parallélisme (mot clé = coeurs)
	}\\

	\hline
    \textbf{Remarques} & \multicolumn{4}{|c|}{
		Le passage dans les rangs lors de l'évolution de l'activité permet un avancement à des vitesses variables pour chaque groupe.
	}\\
	& \multicolumn{4}{|c|}{
		Les grilles communes sont de taille fixe (on ne veut pas d'erreur lors de la mise en commun), les autres sont plus grandes que nécessaires : ils doivent trouver un protocole qui fait attention à la taille
	}\\
	
    
    
	
	
	\hline
\end{tabular}


\end{document}
