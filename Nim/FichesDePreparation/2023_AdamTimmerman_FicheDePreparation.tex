\documentclass[12pt,landscape]{article}
\usepackage[a3paper, margin=0.5cm]{geometry}

\author{Paul ADAM \and Jules TIMMERMAN}
\title{Fiche de préparation :\\Nim}

\begin{document}

\begin{tabular}{|c|c|c|c|c|}
	\hline
	\textbf{Titre} & \multicolumn{4}{|c|}{Jeu de Nim}\\
	\hline
	\textbf{Durée} & \multicolumn{4}{|c|}{55 minutes}\\
	\hline
	\textbf{Prérequis} & \multicolumn{4}{|c|}{Aucun ?}\\
	\hline
	\textbf{Description} & \multicolumn{4}{|c|}{Jeu de Nim : 16 jetons sur la table. Les joueurs à tour de rôle prennent 1,2 ou 3 jetons. Le but est d'être le dernier à prendre un jeton.}\\
	\hline
	\textbf{Matériel} & \multicolumn{4}{|c|}{Jetons de Nim (16 par 2 élèves)}\\
	\hline
	\textbf{Durée} & \textbf{Phase} & \textbf{Activités} & \textbf{Orga} & \textbf{Matériel}\\
	\hline
	
	5' & Explication règles & "On va jouer à un jeu" Partie avec collègue (au tableau) & Tableau & Tableau\\

	1' & Distribution & Paquets déjà fait en avance. Pas sur les tables & Passage rang & .\\
	
	15' & Jeu des enfants & Groupe de deux (pas d'îlots) & Grp de 2 & 16 Carrés\\

	5' & Mise en commun & Demande ce qu'ils ont compris. Partie avec un élève. Montre l'idée & Tableau & .\\

	5' & Explication extension & 16 jetons 1,2,3 OU 4 pièces. Partie de test avec strat & Tableau & .\\

	10' & Jeu des enfants & Groupe de deux & Grp de 2 & Carrés\\

	5' & Remise en commun & Demande ce qu'ils ont compris. Partie avec élève & Tableau & .\\

	5' & InfoParceQue & Stratégie gagante ~ Algo & Tableau & .\\
	
	\hline

	\textbf{Info parce que} & \multicolumn{4}{|c|}{Trouver une stratégie gagnante, c'est savoir comment réussir à coup sûr un objectif. Un ordinateur, on a besoin de lui dire comment "gagner à coup sûr" : c'est un algorithme.
	}\\
	& \multicolumn{4}{|c|}{Un algorithme est une manière de calculer quelque chose. On a besoin que ce soit clair pour pouvoir communiquer avec l'ordinateur qui ne réfléchit pas : il exécute ce qu'on lui dit de faire.}\\
	\hline




														


										



\end{tabular}


\end{document}