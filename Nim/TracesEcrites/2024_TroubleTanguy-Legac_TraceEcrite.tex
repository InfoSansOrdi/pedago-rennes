% Cette trace écrite à été développée par
% + Héloïse Troublé <heloise.trouble@ens-rennes.fr>
% + Erwan Tanguy-Legac <erwan.tanguy-legac@ens-rennes.fr>

% Elle est distribuée sous la licence Creative-Commons by-nc-sa 4.0
% (Attribution, Non-Commercial, Share Alike)
% qui peut être trouvée dans
% [le site de Creative Commons](https://creativecommons.org/licenses/by-nc-sa/4.0/)

\documentclass[12pt,a5paper]{article}
\usepackage[left=1.5cm, right=1.5cm, lines=45, top=0.3in, bottom=0.7in, includehead]{geometry}
\usepackage[french]{babel}
\usepackage[utf8]{inputenc}
\usepackage{fancyhdr}
\usepackage{titling}
\usepackage{changepage}

\usepackage{tikz}
\usetikzlibrary{positioning}

\predate{}
\postdate{}
\date{\vspace{-10ex}}

\lhead{Informatique sans ordi}
\rhead{9 février 2024}
\title{\vspace{-5ex}Jeu de Nim}

\pagenumbering{gobble}

\begin{document}

\maketitle
\thispagestyle{fancy}

\section*{Règles du jeu}
Le jeu de Nim est un jeu qui se joue à deux joueurs, qui jouent chacun leur tour. La partie commence avec 12 jetons sur la table. À chaque tour, le joueur dont c'est le tour peut retirer 1 à 3 jetons. Le joueur qui prend le dernier jeton sur la table gagne la partie.

\section*{Comment gagner ?}
% Les * sont là pour laisser de la place pour que les élèves puissent facilement compléter
On a réfléchi à ce qu'on appelle une \textcolor{white}{stratégie gagnante ******} C'est un ensemble de règles qui nous permet de gagner à tous les coups. Avec ces règles, on s'est rendu compte qu'on pouvait toujours gagner si on jouait en \textcolor{white}{deuxième******}

\begin{figure}[h]
\begin{adjustwidth*}{-2em}{}
\begin{tikzpicture}[inner sep=2mm, rect/.style={rectangle,draw=black,fill=white,thick}]
\node[rect] (origin) {l'adversaire joue};
\node[rect] (1) [above right=of origin] {il a pris 1 jeton\textcolor{white}{s}};
\node[rect] (2) [right=of origin] {il a pris 2 jetons};
\node[rect] (3) [below right=of origin] {il a pris 3 jetons};
\node[rect] (4) [right=of 1] {je prends \textcolor{white}{3 jetons}};
\node[rect] (5) [right=of 2] {je prends \textcolor{white}{2 jetons}};
\node[rect] (6) [right=of 3] {je prends \textcolor{white}{1 jetons}};
\draw [->, thick] (origin) to (1.west);
\draw [->, thick] (origin) to (2.west);
\draw [->, thick] (origin) to (3.west);
\draw [->, thick] (1) to (4);
\draw [->, thick] (2) to (5);
\draw [->, thick] (3) to (6);

\node (7) [right=of 4] {};
\node (8) [right=of 5] {};
\node (9) [right=of 6] {};
\node (p) [below=of 9] {};
\node (q) [below left=of 3] {};
\draw [-, thick] (4) to (7);
\draw [-, thick] (5) to (8);
\draw [-, thick] (6) to (9);
\draw [-, thick] (7.west) to (p.west);
\draw [-, thick] (8.west) to (p.west);
\draw [-, thick] (9.west) to (p.west);

\draw [-, thick] (p.west) to (q.north);
\draw [->, thick] (q.north) to (origin);
\end{tikzpicture}
\end{adjustwidth*}
\end{figure}

Ceci est un algorithme, c'est-à-dire une suite d'instructions précises qu'un ordinateur peut comprendre et reproduire. On a vu avec l'exemple du robot qu'il est nécessaire d'être précis lorsqu'on donne des instructions à l'ordinateur.


\end{document}