\documentclass[10pt]{article}
\usepackage[utf8]{inputenc}
\usepackage[french]{babel}
\usepackage[T1]{fontenc}
\usepackage{mathrsfs}

\usepackage[a4paper,width=280mm,top=10mm,bottom=20mm, landscape]{geometry}

\usepackage{makecell}
\usepackage{booktabs}
\usepackage{array}
\usepackage{subfig}
\usepackage{amsmath,amsfonts,amssymb}
\usepackage{tikz}
\usepackage{tikz-cd}
\usetikzlibrary{shapes}
\usepackage{mathtools}
\usepackage{titlesec} 
\usepackage{enumitem}
\usepackage{listings}
\usepackage{mathdots}
\usepackage{cancel}
\usepackage{comment}

\usepackage{makeidx}

\usepackage{tgpagella}
\makeindex

\usepackage[utf8]{inputenc}

\usepackage{subfiles}

\renewcommand{\arraystretch}{1.5}


\renewcommand\theadalign{c}
\renewcommand\theadfont{\textfont}
\renewcommand\theadgape{\Gape[4pt]}
\renewcommand\cellgape{\Gape[4pt]}

\title{Fiches de préparation}

\author{Misyats Nazar, Regaud Gaëtan}
\date{}
\maketitle

\begin{document}
\begin{center}
\begin{tabular}{|c|>{\hsize=0.3\hsize\centering\arraybackslash}c|c|c|c|}
\hline 
\multicolumn{5}{|c|}{
Gaëtan Regaud \& Misyats Nazar} \\
\hline 
\multicolumn{1}{|l|}{\textbf{Classe} : CM2} & \multicolumn{3}{|l|}{\textbf{Titre} : Les blasons.} & \multicolumn{1}{|l|}{\textbf{Séance n°2}} \\ 
\hline
\multicolumn{5}{|l|}{\textbf{Compétences travaillées} : Communication, élaboration d'un vocabulaire.} \\
\multicolumn{5}{|l|}{\textbf{Objectifs} : Decouvrir les notions de lexique et de langage.} \\
\hline
\textbf{Activité} & \textbf{Script} & \textbf{Notes/Objectifs} & \textbf{Matériel} & \textbf{Temps (min)} \\
\hline 
\thead{Présentation\\de l'activité} & \thead{Introduction devant la classe entière \\ de l'histoire et des consignes.} & \thead{insister sur les règles et sur ce \\ qui est permis et formellement interdit} & \thead{Dessiner un blason au tableau} & \thead{5} \\ 
\hline 
\thead{Explication\\avec exemple} & \thead{Les faire reformuler} & \thead{Pratique avec quelques exemples triviaux \\ pour ne pas donner d'indice.} & \thead{} & \thead{3} \\ 
\hline 
\thead{Distribution\\du matériel} & \thead{A faire après les explications pour ne pas les perdre.} & \thead{} & \thead{Feuilles des blasons vierges et blasons A} & \thead{2} \\ 
\hline 
\thead{Activité} & \thead{Premier essai avec les blasons A} & \thead{Les deux doivent faire deviner et fabriquer. \\ Pas grave s'ils ne finissent pas la feuille} & \thead{} & \thead{10} \\ 
\hline 
\thead{Remise en commun} & \thead{Bilan sur la 1$^{\text{ère}}$ partie\\et sur les problèmes rencontrés.} &  \thead{Demander à des élèves d'expliquer puis \\ introduction des nouveaux blasons B \\ avec la fiche de voc.} &\thead{Blasons B et voc} & \thead{5} \\ 
\hline 
\thead{Remise au travail} & \thead{Extension avec les blasons C.\\Extension en leur proposant d'écrire toutes les \\instructions pour décrire un blason qu'ils ont inventé} & \thead{Les faire améliorer leur vocabulaire \\ pour lever toute ambiguïté} & \thead{Blasons C} & \thead{20} \\ 
\hline 
\thead{Conclusion} & \thead{Bilan et trace écrite} & \thead{Explications récapitulatives.} & \thead{Feuilles à coller pour les élèves.} & \thead{10} \\ 
\hline 
\multicolumn{4}{|l|}{} &
\multicolumn{1}{|c|}{\textbf{Total :} 55} \\
\hline
\end{tabular}
\end{center}
\end{document}
