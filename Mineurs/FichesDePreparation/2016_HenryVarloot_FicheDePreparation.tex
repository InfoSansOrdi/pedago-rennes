\documentclass{article}

\usepackage[utf8]{inputenc}%
\usepackage[francais]{babel}%

\begin{document}
\title{Préparations des interventions Pédago}
\author{Léo Henry et Estelle Varloot}
\date\today
\maketitle
\section{introduction}
Nos deux activités tournaient autour de la notion d'algorithme. Elles étaient
alternées avec celles de Solène et Joshua.

\section{Les mineurs (arbre couvrant minimal d'un graphe)}
Le but de l'activité était de faire découvrir des algorithmes de parcours d'un
arbre couvrant minimal. Nous poussions les élèves à découvrir Kruskal ou Prime.
\paragraph{}
La situation proposée est la suivante: vous disposez d'une équipe de robot et
du plan d'une mine de diamants, dont les galleries sont obstruées par des rochers.
Votre but est de donner des instructions aux robots pour pouvoir récupérer tous
les diamants et retourner à l'entrée, en cassant le moins de pierres possible.
Il faut donc pour cela sélectionner les chemins les moins coûteux à employer.
\paragraph{}
L'activité s'articulait de la manière suivante, avec des temps de repère assez 
indicatifs. Il s'agissait de nos points de repères, mais s'il nous a été utile 
d'en avoir, il l'a aussi été de s'en écarter à 5 minutes près.
\begin{description}
\item[introduction ~10 min] Introduction de l'activité: on va faire de l'informatique
et donc (évidemment) des mineurs, explication de la situation et des règles.
Prenez du temps pour les règles, 5 minutes de plus ici c'est 10 de moins à replacer
les idées au milieu de la séance.
\item[jeu en binômes ~10/15min] Chercher par groupes de deux à trouver la 
solution sur une première mine. Pour ceux qui s'en sortent vite, mieux vaut 
prévoir une seconde carte, à faire en se demandant pourquoi on choisit un chemin.
\item[mise en commun ~5/10min] On reprend le groupe entier et on fait un exemple
au tableau en demandant à chaque étape quel chemin choisir et pourquoi. Explication
de la consigne suivante
\item[travail par 3/4] On reforme des groupes sur une nouvelle carte. Cette fois,
	un élève seulement voit la carte, et ce sont les autres qui lui disent quoi faire.
	C'est le moment de se rendre compte qu'il faut être précis dans sa formulation.
	Le second but est de faire verbaliser l'algorithme (si possible écrire, mais mieux
			vaut ne pas viser une production écrite à tout prix).
	\item[Correction et explication] On remet en commun, et on essaie de faire donner
	la réponse au groupe (avec une correction rédigée sous le coude, Kruskal par exemple
			pour sa simplicité). On fait le lien avec le sujet: c'est bien de l'informatique.
	\end{description}

	\paragraph{}
	Pour moduler l'activité, nous avions prévu des coups de pouces et des aspects
	pour aller plus loin. Les coups e pouces sont simples, ils reviennent juste à se
	poser les bonnes questions: combien ça coûte? est-ce que j'en ai besoin? Attirer
	l'attention des élèves sur ces question est généralement suffisant.
	\\Les questions pour aller plus loin que nous avions trouvées sont, soit
	un petit à-côté (de combien de chemins tu as besoin pour $n$ diamants? par où
			commencer?), soit tout simplement, tu l'as fait comme ça, existe-t'il une autre 
	méthode (et aiguiller sur les bases du second algorithme)? 

	\paragraph{L'explication finale}
	L'explication finale est une partie à bien penser: les élèves sont fatigués, et
	ça peut être ardu. Il est utile d'avoir des mots clés, voir des tournures ou 
	des phrases prévues. Nous nous étions concentrés sur deux points:
	\begin{itemize}
	\item Ça peut servir (avec des exemples: réseau d'électricité ou de téléphone), 
	et pour aller plus loin le graphe. (C'est moins joli mais très général)
	\item Le grand final, brodé sur deux séances : les algorithmes, recettes de cuisine
	/ stratégie à appliquer sans réfléchir, peut s'appliquer à n'importe quelle carte, 
	peut être fait par une machine... 
	\end{itemize}


	\end{document}

