\documentclass[12pt,landscape]{article}
\usepackage[a3paper, margin=0.5cm]{geometry}

\usepackage[french]{babel}
\usepackage[T1]{fontenc}

\author{Paul ADAM \and Jules TIMMERMAN}
\title{Fiche de préparation :\\Ville Embourbée}

\begin{document}

\begin{tabular}{|c|c|c|c|c|}
	\hline
	\textbf{Titre} & \multicolumn{4}{|c|}{Ville Embourbée}\\
	\hline
	\textbf{Durée} & \multicolumn{4}{|c|}{55 minutes}\\
	\hline
	\textbf{Prérequis} & \multicolumn{4}{|c|}{
		Suivre un chemin, compter
	}\\
	\hline
	\textbf{Description} & \multicolumn{4}{|c|}{
		La ville a été victime d'une catastrophe naturelle et tous les chemins ont été détruits. Aidez la mairie à reconstruire en utilisant le moins de pavé possible. (Arbre couvrant minimal)
	}\\
	\hline
	\textbf{Matériel} & \multicolumn{4}{|c|}{
		Jeton de Nim
	}\\
	\hline
	\textbf{Durée} & \textbf{Phase} & \textbf{Activités} & \textbf{Orga} & \textbf{Matériel}\\
	\hline

	5' & Présentation & cf plot plus haut et exemple & Tableau & Tableau\\

	\hline

	15' & Travail & Carte facile. Essayer de trouver une méthode & Groupe de 2 & Carte "facile"\\
	& Pour ceux qui finissent plus vite, leur demander s'ils ont trouvé un algorithme & & &\\

	\hline
	
	5' & Remise en commun & Idée de comment faire ? & Tableau & Tableau\\

	\hline

	5' & Explication & Explication de Kruskal et petite démonstration & Tableau & Tableau\\

	\hline

	15' & Travail & Sur des cartes plus compliquées & Groupe de 2 & Carte plus difficile\\

	\hline

	5' & Remise en commun & Bien réussi ? Des problèmes & Tableau & Tableau\\

	\hline

	5' & InfoParceQue & ... & Tableau & Tableau\\

	\hline


	& \multicolumn{4}{|c|}{
		Arbre couvrant minimal. Algorithme...
	}\\
	\textbf{Info parce que} & \multicolumn{4}{|c|}{
		Justement utile pour minimiser des coûts de connexion dans un réseau (électrique, téléphonique)
	}\\

	
	\hline
\end{tabular}


\end{document}