%%%%%%%%%%%%%%%%%%%%%%%%%%%%% Define Article %%%%%%%%%%%%%%%%%%%%%%%%%%%%%%%%%%
\documentclass[french, a4paper, 11pt]{article}
%%%%%%%%%%%%%%%%%%%%%%%%%%%%%%%%%%%%%%%%%%%%%%%%%%%%%%%%%%%%%%%%%%%%%%%%%%%%%%%

%%%%%%%%%%%%%%%%%%%%%%%%%%%%% Using Packages %%%%%%%%%%%%%%%%%%%%%%%%%%%%%%%%%%
\usepackage[utf8]{inputenc}
\usepackage[T1]{fontenc}
\usepackage{lmodern}
\usepackage{graphicx}
\usepackage{amssymb}
\usepackage{amsmath}
\usepackage{amsthm}
\usepackage{empheq}
\usepackage{mdframed}
\usepackage{booktabs}
\usepackage{lipsum}
\usepackage{graphicx}
\usepackage{color}
\usepackage{psfrag}
\usepackage{pgfplots}
\usepackage{bm}
\usepackage{tikz}
\usepackage{babel}
\usepackage[left=2cm, right=2cm, top = 2cm, bottom = 2cm]{geometry}
%%%%%%%%%%%%%%%%%%%%%%%%%%%%%%%%%%%%%%%%%%%%%%%%%%%%%%%%%%%%%%%%%%%%%%%%%%%%%%%

% Other Settings

%%%%%%%%%%%%%%%%%%%%%%%%%% Page Setting %%%%%%%%%%%%%%%%%%%%%%%%%%%%%%%%%%%%%%%
\geometry{a4paper}


%%%%%%%%%%%%%%%%%%%%%%%%%%%%%%% Title & Author %%%%%%%%%%%%%%%%%%%%%%%%%%%%%%%%
\title{Fiche de préparation, crêpier psychorigide}
\author{Andrieux Martin, Vitry Thomas}
\date{}
%%%%%%%%%%%%%%%%%%%%%%%%%%%%%%%%%%%%%%%%%%%%%%%%%%%%%%%%%%%%%%%%%%%%%%%%%%%%%%%

\begin{document}
    Aujourd'hui, nous avons assemblé des crocodiles et leurs oeufs. Avec certaine règles, il est possible d'utiliser ces assemblages pour représenter des choses plus compliqués comme des formules logique (le vrai, faux, pas vrai \dots) voire même des calculs avec des assemblages encore plus gros.
    
    En vérité, cette méthode de calcul s'appelle le \textbf{lambda calcul}, elle est très simple à appliquer une fois l'assemblage construit et est suffisement puissante pour pouvoir fabriquer des ordinateurs qui fonctionneraient avec.\\

    Aujourd'hui, nous avons assemblé des crocodiles et leurs oeufs. Avec certaine règles, il est possible d'utiliser ces assemblages pour représenter des choses plus compliqués comme des formules logique (le vrai, faux, pas vrai \dots) voire même des calculs avec des assemblages encore plus gros.
    
    En vérité, cette méthode de calcul s'appelle le \textbf{lambda calcul}, elle est très simple à appliquer une fois l'assemblage construit et est suffisement puissante pour pouvoir fabriquer des ordinateurs qui fonctionneraient avec.\\

    Aujourd'hui, nous avons assemblé des crocodiles et leurs oeufs. Avec certaine règles, il est possible d'utiliser ces assemblages pour représenter des choses plus compliqués comme des formules logique (le vrai, faux, pas vrai \dots) voire même des calculs avec des assemblages encore plus gros.
    
    En vérité, cette méthode de calcul s'appelle le \textbf{lambda calcul}, elle est très simple à appliquer une fois l'assemblage construit et est suffisement puissante pour pouvoir fabriquer des ordinateurs qui fonctionneraient avec.\\

    Aujourd'hui, nous avons assemblé des crocodiles et leurs oeufs. Avec certaine règles, il est possible d'utiliser ces assemblages pour représenter des choses plus compliqués comme des formules logique (le vrai, faux, pas vrai \dots) voire même des calculs avec des assemblages encore plus gros.
    
    En vérité, cette méthode de calcul s'appelle le \textbf{lambda calcul}, elle est très simple à appliquer une fois l'assemblage construit et est suffisement puissante pour pouvoir fabriquer des ordinateurs qui fonctionneraient avec.\\

    Aujourd'hui, nous avons assemblé des crocodiles et leurs oeufs. Avec certaine règles, il est possible d'utiliser ces assemblages pour représenter des choses plus compliqués comme des formules logique (le vrai, faux, pas vrai \dots) voire même des calculs avec des assemblages encore plus gros.
    
    En vérité, cette méthode de calcul s'appelle le \textbf{lambda calcul}, elle est très simple à appliquer une fois l'assemblage construit et est suffisement puissante pour pouvoir fabriquer des ordinateurs qui fonctionneraient avec.\\

    Aujourd'hui, nous avons assemblé des crocodiles et leurs oeufs. Avec certaine règles, il est possible d'utiliser ces assemblages pour représenter des choses plus compliqués comme des formules logique (le vrai, faux, pas vrai \dots) voire même des calculs avec des assemblages encore plus gros.
    
    En vérité, cette méthode de calcul s'appelle le \textbf{lambda calcul}, elle est très simple à appliquer une fois l'assemblage construit et est suffisement puissante pour pouvoir fabriquer des ordinateurs qui fonctionneraient avec.\\

    Aujourd'hui, nous avons assemblé des crocodiles et leurs oeufs. Avec certaine règles, il est possible d'utiliser ces assemblages pour représenter des choses plus compliqués comme des formules logique (le vrai, faux, pas vrai \dots) voire même des calculs avec des assemblages encore plus gros.
    
    En vérité, cette méthode de calcul s'appelle le \textbf{lambda calcul}, elle est très simple à appliquer une fois l'assemblage construit et est suffisement puissante pour pouvoir fabriquer des ordinateurs qui fonctionneraient avec.\\

\end{document}