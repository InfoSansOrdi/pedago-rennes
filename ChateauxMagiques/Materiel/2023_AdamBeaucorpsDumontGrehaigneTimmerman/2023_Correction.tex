\documentclass{article}

\usepackage[french]{babel}
\usepackage[T1]{fontenc}
\usepackage[a4paper, margin=2cm]{geometry}
\usepackage{graphicx}
\usepackage[inkscapeformat=png]{svg}
\usepackage{svg}

\usepackage{tikz}
\usetikzlibrary{automata, arrows.meta, positioning}

\newcounter{question}
\newenvironment{question}{
		\noindent\stepcounter{question}\thequestion. 
	}{
		\vspace{10pt}
	}


\begin{document}
 \centering\section*{Correction chateau magique}
 \vspace{1cm}
\begin{minipage}[t]{0.45\textwidth}
	\begin{center}
		\textbf{Petit Château}
	\end{center}

	\begin{question}
		Est-ce qu'il existe une formule	qui mène à la salle du trône ? Si oui, laquelle ? Si non, pourquoi ?

		\texttt{(cadabra)(abra)(cadabra)}
	\end{question}

	\begin{question}
		Le magicien aimerait effectuer exactement 6 déplacements. Est-ce possible ? Si oui, quelle est la formule ? Si non, pourquoi ?

		Au moins une boucle \texttt{cadabra}
	\end{question}

	\begin{question}
		Est-ce qu'il existe une formule plus longue que toutes les autres ? Si oui laquelle ? Si non, pourquoi ?

		Non. Si tu en prends une, tu peux en trouver une plus grande.
	\end{question}

	\begin{question}
		Est-ce qu'il y a un endroit à ne pas visiter ? Si oui, lequel ? Pourquoi ?

		Les oubliettes (on ne peut pas ressortir)
	\end{question}

	\begin{question}
		Est-ce qu'il existe une formule qui commence par abracabra et qui mène au trône ? Si oui, laquelle ? Si non, pourquoi ?

		Non.
	\end{question}
\end{minipage}
\hfill
\setcounter{question}{0}
\begin{minipage}[t]{0.45\textwidth}
	\begin{center}
		\textbf{Grand Château}
	\end{center}

	\begin{question}
		Est-ce qu'il existe une formule qui mène à la salle du trône en passant par la salle du trésor ? Si oui, laquelle ? Si non, pourquoi ?

		Oui (ex : Ecurie $\rightarrow$ Trésor $\rightarrow$ trône)
	\end{question}

	% \begin{question}
	% 	Est-ce qu'il existe une formule qui mène à la salle du trône en passant par la grande salle du banquet ? Si oui laquelle ? Si non, pourquoi ?


	% \end{question}

	\begin{question}
		Est-ce qu'il y a un endroit à ne pas visiter ? Si oui, lequel ? Si non, pourquoi ?

		Salle des gardes
	\end{question}

	\begin{question}
		Est-ce qu'il existe une formule la plus courte ? Si oui laquelle ? Si non, pourquoi ?

		Oui. \texttt{cada-ra-bra-da-bra-braca} (19 lettres). Différent du moins d'état \texttt{cadabra-cadabra-bra-cadabra} (24 lettres)
	\end{question}

	\begin{question}
		Quelle formule permet de passer par un nombre maximal de salle ? (sans ressortir de la salle du trône)

		Passer par : Ronde $\rightarrow$ Magicien $\rightarrow$ Vin$\rightarrow$ Grenier $\rightarrow$ Jardin $\rightarrow$ (Vin)$\rightarrow$ Cuisine $\rightarrow$ Banquet
	\end{question}

	\begin{question}
    Une même formule peut-elle nous mener à deux endroits différents ?

    Pas ici. Exemple de problème :

    \begin{tikzpicture}[node distance = 2.5cm, on grid, auto]
        \node (1) [state, initial, initial text= ] {1};
        \node (2) [state, right = of 1] {2};
        \node (3) [state, right = of 2] {3};
        \node (4) [state, below = of 1] {4};

        \path[-stealth,thick]
            (1) edge node {abra} (2)
            (2) edge node {cadabra} (3)

            (1) edge node {abracadabra} (4);

        
    \end{tikzpicture}
	\end{question}
\end{minipage}

\vspace{1cm}

\centering\subsection*{Supplément}
-> Construire un automate avec un nombre pair de "abra" et un nombre impair de "cada".

Vous commencez au pont-levis, passez dans les salles de votre choix, et finissez dans la salle du trône. 
Vous pouvez passer par la salle du trône sans vous y arrêter.

\vspace{1cm}

\includesvg{2023_Supplement.svg}

 
\end{document}